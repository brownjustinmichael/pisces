%!TEX root = /Users/justinbrown/Dropbox/pisces/documentation/publication.tex
\subsection{Equations} % (fold)
\label{sub:equations}

	The Boussinesq equations for a fluid with density contributions from temperature but not composition are as follows:
	\begin{align}
		\frac{\partial{\vel}}{\partial{t}}+\vel\cdot\nabla\vel&=-\nabla\p+\frac{\alpha\grav}\T\zunit+\visc\nabla^{2}\vel\\
		\frac{\partial{\T}}{\partial{t}}+\vel\cdot\nabla\T-w\Tad&=\nabla\cdot\left(\kt\left(1+\phi\left(\comp\right)\right)\nabla\T\right)\\
		\frac{\partial{\comp}}{\partial{t}}+\vel\cdot\nabla\comp&=\kc\nabla^{2}\comp\\
		\nabla\cdot\vel&=0
	\end{align}
		For application to the Sun, $\comp$ is a scalar tracer for the ionization fraction, which we assume can be advected and diffuses according to Fick's Law.
		We note here that for the case of the Sun, it would be more physical to have $\phi$ depend on the temperature rather than a scalar tracer; however, this would make the boundary conditions somewhat less well-posed.
			For the moment, we simply state that $\kc$ is expected to be much larger than microscopic diffusion of ions.	

	We apply the boundary conditions $\comp\left(z=-[L]/2\right)=1/2$, $\comp\left(z=[L]/2\right)=-1/2$, where $[L]$ is the height of the domain, since the zero point of this tracer is irrelevant.
		We let $\T\left(z=-[L]/2\right)=0$ as an arbitrary choice and $\partial \T/\partial z\left(z=[L]/2\right)=\partial \T/\partial z|_{\mathrm{CZ}}$, which is an assumed constant input.
		We assume the top and bottom are impermeable and stress-free.
		All variables are assumed horizontally periodic.

	We define our non-dimensional equations as follows, where we have taken $\nu=[L]^{2}/[t]$, $[T]=[L]|\partial \T/\partial z|_{\mathrm{CZ}}-\Tad|$, etc.

	\begin{align}
		\frac{\partial{\ndvel}}{\partial{\ndt}}+\ndvel\cdot\ndnabla\ndvel&=-\ndnabla\ndp+\frac{\Ra_{\mathrm{CZ}}}{\Pran_{\mathrm{CZ}}}\ndT\zunit+\ndnabla^{2}\ndvel\\
		\frac{\partial{\ndT}}{\partial{\ndt}}+\ndvel\cdot\ndnabla\ndT+\ndw\left(\frac{1}{\chi}-1\right)&=\ndnabla\cdot\left(\Pran^{-1}_{\mathrm{CZ}}\left(1+\phi\left(\comp\right)\right)\ndnabla\ndT\right)\\
		\frac{\partial{\comp}}{\partial{\ndt}}+\ndvel\cdot\ndnabla\comp&=\Sc^{-1}\ndnabla^{2}\comp\\
		\ndnabla\cdot\ndvel&=0
	\end{align}

	The value $\chi$ is defined as
	\begin{equation}
		\chi=\left|\frac{\partial \T/\partial z|_{\mathrm{CZ}}-\Tad}{\partial \T/\partial z|_{\mathrm{CZ}}}\right|.
	\end{equation}
	This makes the top boundary condition $\partial \ndT/\partial \ndz\left(z=[L]/2\right)=-1/\chi$.
	We define $\Pran_{\mathrm{CZ}}$ and $\Ra_{\mathrm{CZ}}$ to be the Prandtl number and Rayleigh number in the convection zone, respectively, by enforcing that $\phi\left(\comp\right)$ be $0$ for $\comp=-1/2$ and everywhere greater than or equal to zero.
		If the desire is to set the stiffness of the system, $S=|\Ra_{RZ}/\Ra_{CZ}$, where we are being somewhat loose with the definition of the Rayleigh number as the radiative fluid is stable to convection.
		For a given $\chi$ and a given $S$, the maximum $\phi$ is given by the analogous equation in Korre et al. (in prep) by solving
		\begin{equation}
			\Pran_{\mathrm{RZ}}^{2}-\left(1-\chi\right)\Pran_{\mathrm{CZ}}\Pran_{\mathrm{RZ}}-\Pran_{\mathrm{CZ}}^{2}\chi S=0
		\end{equation}
		for $\Pran_{\mathrm{RZ}}$, and then $\phi_{\mathrm{max}}=\left(\Pran_{\mathrm{RZ}}-\Pran_{\mathrm{CZ}}\right)/\Pran_{\mathrm{CZ}}$.

	For the time being, we are using a hyperbolic tangent function, so
	\begin{equation}
		\phi\left(\comp\right)=\frac{\phi_{\mathrm{max}}}{2}\tanh{\frac{\comp-\comp_{0}}{d_{\comp}}}+\frac{\phi_{\mathrm{max}}}{2},
	\end{equation}
	where $\comp_{0}$ and $d_{\comp}$ are free parameters.
		This is intended to represent the opacity of material with some ionization state, which is very high above some threshold and very low below it.

% subsection equations (end)

\subsection{Additional Methods} % (fold)
\label{sub:additional_methods}

	We must introduce one additional method for the treatment of this new non-linear term.

% subsection additional_methods (end)

\subsection{Linear Stability} % (fold)
\label{sub:linear_stability}

	We first want to subtract the background hydrostatic state of each equation, beginning with the composition equation:
	\begin{equation}
		0=\ndnabla^{2}\comp,
	\end{equation}
	and we will assume a background state of $\comp=-\ndz$, which satisfies our two boundary conditions.
	Next, we look at $\T$:
	\begin{align}
		0&=\ndnabla\cdot\left(\left(1-\phi z\right)\ndnabla\ndT\right),\\
		0&=\left(1-\phi z\right)\frac{\partial^{2}}{\partial \ndz^{2}}\ndT-\phi\frac{\partial}{\partial \ndz}\ndT,\\
	\end{align}
	to which the solution that satisfies the boundaries is $\ndT=\frac{\psi}{\phi}\ln{\frac{1-\phi}{1-\phi \ndz}}$.
	The solution to the last equation is trivial and can just be absorbed into the background pressure.

	Assuming a solution of the form $\ndT=\ndT\left(z\right)e^{\lambda \ndt + i k \ndx}$ and ignoring non-linear terms, the equations become:
	\begin{align}
		\lambda\ndvel&=-ik\ndp\xunit-\frac{\partial}{\partial \ndz}\ndp\zunit+\frac{\Ra}{\Pran}\ndT\zunit+\left(-k^{2}+\frac{\partial^{2}}{\partial \ndz^{2}}\right)\ndvel\\ 
		\lambda\ndT+\ndw\left(1 + \frac{\psi}{1-\ndz\phi}\right)&=\Pran^{-1}\left(1+\phi\comp-\phi z\right)\nabla^{2}\left(\ndT+\frac{\psi}{\phi}\ln{\frac{1-\phi}{1-\phi \ndz}}\right)-\Pran^{-1}\phi\frac{\partial}{\partial \ndz}\ndT+\Pran^{-1}\frac{\phi\psi}{1-\ndz\phi}\frac{\partial}{\partial \ndz}\xi\\
		&=\Pran^{-1}\left(1-\phi z\right)\left(-k^{2}+\frac{\partial^{2}}{\partial \ndz^{2}}\right)\ndT+\Pran^{-1}\phi\comp\frac{\psi\phi}{\left(1-\phi \ndz\right)^{2}}-\Pran^{-1}\phi\frac{\partial}{\partial \ndz}\ndT+\Pran^{-1}\frac{\phi\psi}{1-\ndz\phi}\frac{\partial}{\partial \ndz}\xi\\
		\lambda\comp-\ndw\ndz&=\frac{\Le}{\Pran}\left(-k^{2}+\frac{\partial^{2}}{\partial \ndz^{2}}\right)\comp\\
		ik\ndu + \frac{\partial}{\partial \ndz}\ndw&=0
	\end{align}

	This can be arranged into one eighth order ODE, but even writing it down would take pages, and I've yet to find a solution.

% subsection linear_stability (end)