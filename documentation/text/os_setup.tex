%!TEX root = /Users/justinbrown/Dropbox/pisces/documentation/publication.tex
\subsection{Equations} % (fold)
\label{sub:equations}

	We define our non-dimensional equations as follows, where we have taken $\nu=[L]^{2}/[t]$, $[T]=[L]|\Tad|$.
	We let $[L]$ be half the height of the domain, going from $-[L]$ to $[L]$.
	$\flux$ is the incident flux on the bottom boundary.

	\begin{align}
		\frac{\partial{\ndvel}}{\partial{\ndt}}+\ndvel\cdot\ndnabla\ndvel&=-\ndnabla\ndp+\frac{\Ra}{\Pran}\ndT\zunit+\ndnabla^{2}\ndvel\\
		\frac{\partial{\ndT}}{\partial{\ndt}}+\ndvel\cdot\ndnabla\ndT+\ndw&=\ndnabla\cdot\left(\left(\Pran^{-1}+\Pran_{\T}^{-1}\ndT+\Pran_{\comp}^{-1}\comp\right)\ndnabla\ndT\right)\\
		\frac{\partial{\comp}}{\partial{\ndt}}+\ndvel\cdot\ndnabla\comp&=\frac{\Le}{\Pran}\ndnabla^{2}\comp\\
		\ndnabla\cdot\ndvel&=0
	\end{align}

	Some decisions need to be made regarding definitions.
	We choose to allow $\comp$ to go from -1 to 1, allowing us to define $\Pran$ to be something like the average Prandtl number.
	We will require (for the moment) that $\Pran_{\comp}^{-1}>0$, so $\comp$ must go to 1 in the radiative zone (stronger diffusion, weaker convection) and to -1 in the convective zone.
	This sets our boundary conditions for $\comp$.
	We will also define $\ndTad\equiv\Tad\visc/\flux$ as the non-dimensional adiabatic temperature gradient.

	We want the system to be generated such that the temperature gradient for pure diffusion and no contributions to the diffusion coefficient from composition be exactly the adiabatic temperature gradient.
		This constitutes $\ndTad=-\Pran$ as the stable temperature gradient is simply $\ndnabla\ndT=-F\nu/F\kt=-\Pran$.
		We'll let $\ndTad=-\psi\Pran$ for some added flexibility.
	Unfortunately, this choice complicates our bottom boundary.
		Since we are requiring the flux through that boundary be $\flux$, the gradient must be $\ndnabla\ndT=-\psi\left(\Pran^{-1}+\Pran_{\T}^{-1}\ndT+\Pran_{\comp}^{-1}\comp\right)^{-1}$.
		This is of course, a problem since we would have to prescribe $\T$ at the bottom boundary, so for the moment, we investigate only systems with $\Pran_{\T}^{-1}=0$.
		The boundary condition is then $\ndnabla\ndT=-\psi\left(\Pran^{-1}+\Pran_{\comp}^{-1}\right)^{-1}$ at the bottom boundary.
		We may choose the $\T$ of the top boundary arbitrarily, so we let it be 0.

	The boundaries are impermeable and stress-free as well.

	% We first want to subtract the background hydrostatic state:
	% \begin{equation}
	% 	0=\ndnabla^{2}\comp,
	% \end{equation}
	% and we will assume a background state of $\comp=-\ndz$.
	% Next, we look at $\T$:
	% \begin{align}
	% 	0&=\ndnabla\cdot\left(\left(\Pran^{-1}-\Pran_{\comp}^{-1}z\right)\ndnabla\ndT\right),\\
	% 	0&=-\Pran_{\comp}^{-1}\comp_{0z}\frac{\partial^{2}}{\partial \ndz^{2}}\ndT+\left(\Pran^{-1}-\Pran_{\comp}^{-1}\ndz\right)\frac{\partial^{2}}{\partial \ndz^{2}}\ndT,\\
	% \end{align}
	% to which a solution is $\ndT=-\frac{F}{\Pran}\ndz$.
	% The solution to the last equation is trivial and can just be absorbed into the background pressure.

	% Assuming a solution of the form $\ndT=\ndT\left(z\right)e^{\lambda \ndt + i k \ndx}$ and ignoring non-linear terms, the equations become:
	% \begin{align}
	% 	\lambda\ndvel&=-ik\ndp\xunit-\frac{\partial}{\partial \ndz}\ndp\zunit+\frac{\Ra}{\Pran}\ndT\zunit+\left(-k^{2}+\frac{\partial^{2}}{\partial \ndz^{2}}\right)\ndvel\\
	% 	\lambda\ndT+\ndw\left(\Pran - \frac{F}{\Pran}\right)&=\left(\Pran^{-1}-\Pran^{-1}_{\comp}z\right)\left(-k^{2}+\frac{\partial^{2}}{\partial \ndz^{2}}\right)\ndT-\frac{F}{\Pran}\frac{\partial}{\partial \ndz}\ndT - \frac{\partial}{\partial \ndz}\comp\\
	% 	\lambda\comp-\ndw\ndz&=\frac{\Le}{\Pran}\left(-k^{2}+\frac{\partial^{2}}{\partial \ndz^{2}}\right)\comp\\
	% 	ik\ndu + \frac{\partial}{\partial \ndz}\ndw&=0
	% \end{align}

	The Rayleigh number and Prandtl number are well-defined for a fluid with $\comp=0$, but what are they otherwise?
		In this context, the Rayleigh number in the equation is defined as $\Ra=\alpha g \flux H^{4}/\visc/\kt^{2}$, which is the Rayleigh number in the absence of fluid motion.
			The full $\Ra$ (including the adiabatic temperature gradient) is $\Ra=\alpha g \Tad \left(\psi/\left(1 + \phi\comp\right) - 1\right) H^{4}/\visc\kt\left(1+\phi\comp\right)$.
			The Rayleigh number at the top of the convection zone is then this with $\comp=-1$ and that at the bottom boundary has $\comp=1$.
		The stiffness of the simulation is then $S=\frac{\Ra\left(1\right)}{\Ra\left(-1\right)}$, which is
		\begin{equation}
			S=\frac{\left(\psi/\left(1 + \phi\right) - 1\right)\left(1-\phi\right)}{\left(\psi/\left(1 - \phi\right) - 1\right)\left(1+\phi\right)}
		\end{equation}
		\begin{equation}
			S=\frac{\left(\psi - 1 - \phi\right)\left(1-\phi\right)^{2}}{\left(\psi - 1 + \phi\right)\left(1+\phi\right)^{2}}
		\end{equation}
		\begin{equation}
			S=\frac{\psi\left(1-2\phi+\phi^{2}\right)-1+2\phi^{2}-\phi^{4}}{\psi\left(1+2\phi+\phi^{2}\right)-1+2\phi^{2}-\phi^{4}}.
		\end{equation}




% subsection equations (end)