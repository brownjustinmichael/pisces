\section{Model} % (fold)
\label{sec:model}

\subsection{Pseudo-Incompressible Equations} % (fold)
\label{sub:pseudo}

We start from the fully compressible equations of hydrodynamics in an irrotational system without magnetic fields (e.g. \citet{Braginsky2006}),
\begin{subequations}
	\begin{align}
		\label{eq:velocity}
		\rho\frac{D}{Dt}\vect{u}&=-\nabla p-\rho\nabla\Phi+\nabla\cdot\vect{\Pi},\\
		\label{eq:density}
		\frac{D}{Dt}\rho&=-\rho\nabla\cdot\vect{u},\\
		\label{eq:entropy}
		\rho T \frac{D}{Dt}s&=\nabla\vect{u}:\vect{\Pi}+\sum_{i}\mu_{i}\nabla\cdot\vect{C}_{i}-\nabla\cdot{\vect{H}},\\
		\label{eq:composition}
		\rho\frac{D}{Dt}\xi_{i}&=-\nabla\cdot\vect{C}_{i},
	\end{align}
\end{subequations}
where $\rho$ is the mass density, $\vect{u}$ is the velocity, $\Phi$ is the gravitational potential energy, $s$ is the entropy, and $\xi_{i}$ is the fractional abundance of the $i$th compositional component of the fluid.

Given an equation of state of the form $e(\rho,s,\xi_{1},\xi_{2},\dots)$, we can formally define 
\begin{subequations}
	\begin{align}
		p\equiv\rho^{2}\frac{\partial e}{\partial \rho},\\
		T\equiv\frac{\partial e}{\partial s},\\
		\mu_{i}\equiv\frac{\partial e}{\partial \xi_{i}},
	\end{align}
\end{subequations}
which are the pressure, temperature, and chemical potentials, respectively.
The remaining quantities, $\vect{Pi}$, $\vect{H}$, and $\vect{C_{i}}$ are the stress tensor, heat flux vector, and compositional flux vectors, respectively.

These equations as they are conserve momentum through Equation \ref{eq:velocity}, mass through Equation \ref{eq:density}, and energy as follows:
\begin{equation}
	\rho\frac{D}{Dt}\left(\vect{u}^{2}/2+e+\Phi\right)+\nabla\cdotp\vect{u}=\nabla\cdot\left(\Pi\cdot\vect{u}-\vect{H}\right),
\end{equation}
which can be derived trivially using Equations \ref{eq:velocity} through \ref{eq:composition} and the first law of thermodynamics:
\begin{equation}
	\label{eq:firstlaw}
	de=Tds+\frac{p}{\rho^{2}}d\rho+\sum_{i}\mu_{i}d\xi_{i}.
\end{equation}

\citet{Vasil2013} show that in the absence of any nonconservative effects, such as diffusion, and under the approximation that the fluid is constrained to be in pressure balance, the pseduo-incompressible equations take the following form:
\begin{subequations}
	\begin{align}
		\rho\frac{D}{Dt}\vect{u}&=-\nabla p_{0}-p_{0}^{\frac{c_{p}}{c_{V}}}\nabla\left(\frac{p_{1}}{p_{0}^{\frac{c_{p}}{c_{V}}}}\right)-\rho\nabla\Phi,\\
		\frac{D}{Dt}\rho&=-\rho\nabla\cdot\vect{u},\\
		\nabla\cdot p_{0}^{\frac{c_{p}}{c_{V}}}\vect{u}&=0,
	\end{align}
\end{subequations}
where $p_{0}$ is a nonuniform but static reference pressure, $p_{1}$ is a small pressure perturbation, and $c_{V}$ and $c_{p}$ are the specific heat capacities at constant volume and pressure, respectively.
As an additional assumption to arrive at this expression, it must be assumed that $\frac{c_{p}}{c_{V}}=\frac{\rho}{p}\left.\frac{\partial p}{\partial \rho}\right|_{s}$ is constant, which is the case for an ideal gas.
It should be noted that these equations do not take the same form as those from the original ones in \citet{Durran1989}, rather they are derived to conserve mass and energy.

Wood (private communication) has used these results to derive a set of equations with varying composition and allowing for diffusion while still conserving energy and mass:
\begin{subequations}
	\begin{align}
		\label{eq:pieos}
		p_{0}\left(\vect{x}\right)&=p\left(\rho,s,\xi_{1},\xi_{2},\dots\right),\\
		\label{eq:pivelocity}
		\rho\frac{D}{Dt}\vect{u}&=-\nabla\left(p_{0}+p_{1}\right)+\frac{p_{1}}{\rho}\left(\frac{\partial p}{\partial \rho}\right)^{-1}\nabla p_{0}-\rho\nabla\Phi+\nabla\cdot\vect{\Pi},\\
		\label{eq:pidensity}
		\frac{D}{Dt}\rho&=-\rho\nabla\cdot\vect{u},\\
		\label{eq:pientropy}
		\rho T \frac{D}{Dt}s&=\nabla\vect{u}:\vect{\Pi}+\sum_{i}\mu_{i}\nabla\cdot\vect{C}_{i}-\nabla\cdot{\vect{H}},\\
		\label{eq:picomposition}
		\rho\frac{D}{Dt}\xi_{i}&=-\nabla\cdot\vect{C}_{i},
	\end{align}
\end{subequations}
where the quantities $T$ and $\mu_{i}$ are still the ``true'' values of temperature and chemical potential, but due to the approximation of the equation of state by fixing the pressure, these may no longer be the derivatives of $e$.
He has shown that to conserve energy and mass,
\begin{subequations}
	\begin{align}
		\label{eq:realT}
		T&=\frac{\partial e}{\partial s}+p_{1}\frac{\partial \frac{\partial e}{\partial s}/\partial\rho}{\partial p/\partial\rho},\\
		\label{eq:realmu}
		\mu_{i}&=\frac{\partial e}{\partial \xi_{i}}+p_{1}\frac{\partial \frac{\partial e}{\partial \xi_{i}}/\partial\rho}{\partial p/\partial\rho}.
	\end{align}
\end{subequations}
This necessitates the use of an equation of state.
% subsection pseudo (end)

\subsection{Equation of State} % (fold)
\label{sub:eos}

The nature of the equations derived in \ref{sub:pseudo} requires an equation of state to determine the true temperature, $T$, and chemical potential, $\mu$.
Unfortunately, unlike in the original \citet{Durran1989} formulation of the pseudo-incompressible equations, we cannot assume an ideal gas as our system has multiple chemical components.
There does exist a formulation, known as an ideal mixture or solution XX REFERENCE XX, which describes a gas with multiple ideal components.
In particular, this system has the properties that $n$, the total number density, is related to the component number densities, $n_{i}$, and fractional abundance, $\xi_{i}$ by
\begin{equation}
	n_{i}=\xi_{i}n,
\end{equation}
where we require the fractional abundances to sum to unity.
From this, we can derive the relationship between the mass densities:
\begin{equation}
	\frac{\rho_{i}}{m_{i}}=\xi_{i}\frac{\rho}{\overline{m}},
\end{equation}
where $m_{i}$ is the mass of one particle in the corresponding component, and $\overline{m}$ is the mean particle weight in the system, weighted by the fractional abundances.

Recall that the specific energy of an ideal gas can be expressed as
\begin{equation}
	e\left(\rho,s\right)=\frac{c_{V}}{m}\left(\frac{\rho}{m}\phi e^{\frac{ms}{k}}\right)^{\frac{k}{c_{V}}},
\end{equation}
where $k$ is the Boltzmann constant, $c_{V}$ is the specific heat capacity at constant volume ($3k/2$ for a perfect monatomic gas), $m$ is the particle mass, $s$ is the specific entropy, $\phi$ is a constant but can vary for different gases.
In an ideal mixture, we can add the specific energies as follows:
\begin{align}
	\overline{m} e\left(\rho,s_{1},s_{2},\dots,\xi_{1},\xi_{2},\dots\right)&=\sum_{i} m_{i} \xi_{i} e_{i}\left(\rho,s_{i},\xi_{i}\right),\\
	&=\sum_{i}c_{V}\left(\frac{\rho\xi_{i}}{m_{i}}\phi_{i} e^{\frac{m_{i}s_{i}}{k}}\right)^{\frac{k}{c_{V}}},
\end{align}
where subscripted quantities are specific to gas components.
Given that we have evolution equations for the density, fractional abundances, and total entropy, we'd like to have our equation of state as a function of only those quantities.
For an ideal mixture, we can relate the specific entropies as follows:
\begin{equation}
	\label{eq:totalentropy}
	\overline{m} s=\sum_{i}m_{i}\xi_{i}s_{i}-k\sum_{i}\xi_{i}\ln{\xi_{i}}.
\end{equation}
The last summation is known as the entropy of mixing and comes from the increased number of possible states due to the introduction of non-identical particles.

This unfortunately is insufficient to express the internal energy solely in terms of the global entropy; we must make an additional approximation.
We choose to assume that the temperature of the individual components, $T_{i}\equiv\partial e_{i}/\partial s_{i}$, are the same.
While this assumption does not usually hold in systems with few collisions (e.g. the interstellar medium), it is typically a good approximation in gases with frequent collisions.
\begin{align}
	T_{i}&=\frac{\partial e_{i}}{\partial s_{i}},\\
	&=\left(\frac{\rho\xi_{i}\phi_{i}}{\overline{m}}e^{\frac{m_{i}s_{i}}{k}}\right)^{\frac{k}{c_{V}}},
\end{align} 
so for all the $T_{i}$ to be equal,
\begin{equation}
	\left(\xi_{i}\phi_{i}e^{\frac{m_{i}s_{i}}{k}}\right)=\left(\xi_{j}\phi_{j}e^{\frac{m_{j}s_{j}}{k}}\right).
\end{equation}
In a two-component gas, we can use this to relate the entropies by using Equation \ref{eq:totalentropy}:
\begin{align}
	e^{\frac{m_{1}s_{1}}{k}}&=\left(\frac{\xi_{2}\phi_{2}}{\xi_{1}\phi_{1}}\right)^{\xi_{2}}e^{\frac{\overline{m}s}{k}+\xi_{1}\ln{\xi_{1}}+\xi_{2}\ln{\xi_{2}}},\\
	e^{\frac{m_{2}s_{2}}{k}}&=\left(\frac{\xi_{1}\phi_{1}}{\xi_{2}\phi_{2}}\right)^{\xi_{1}}e^{\frac{\overline{m}s}{k}+\xi_{1}\ln{\xi_{1}}+\xi_{2}\ln{\xi_{2}}}.
\end{align}
We can thus express the specific internal energies as
\begin{align}
	e_{1}\left(\rho,s,\xi_{1},\xi_{2}\right)&=c_{V}\left(\frac{\rho\xi_{1}\phi_{1}}{m_{1}}\left[\frac{\xi_{2}\phi_{2}}{\xi_{1}\phi{1}}\right]^{\xi_{2}} e^{\frac{\overline{m}s}{k}+\xi_{1}\ln{\xi_{1}}+\xi_{2}\ln{\xi_{2}}}\right)^{\frac{k}{c_{V}}},\\
	e_{2}\left(\rho,s,\xi_{1},\xi_{2}\right)&=c_{V}\left(\frac{\rho\xi_{2}\phi_{2}}{m_{2}}\left[\frac{\xi_{1}\phi_{1}}{\xi_{2}\phi{2}}\right]^{\xi_{1}} e^{\frac{\overline{m}s}{k}+\xi_{1}\ln{\xi_{1}}+\xi_{2}\ln{\xi_{2}}}\right)^{\frac{k}{c_{V}}},\\
	\overline{m} e\left(\rho,s,\xi_{1}\right)&=\sum_{i} m_{i} \xi_{i} e_{i}\left(\rho,s,\xi_{1},1-\xi_{1}\right).
\end{align}
It can be shown nontrivially that the derived pressure from this formulation, $p\equiv\rho^{2}\partial e /\partial \rho$, is
\begin{equation}
	p=\frac{k\rho T}{\overline{m}},
\end{equation}
where $T\equiv\partial e/\partial s$, which is what one would expect na\"ively of an ideal gas with multiple components.

With some calculus, we can derive useful expressions for Equations \ref{eq:realT} and \ref{eq:realmu}:
\begin{subequations}
	\begin{align}
		T&=\frac{\partial e}{\partial s}+p_{1}\frac{\overline{m}}{\left(c_{V}+k\right)\rho},\\
		\mu_{i}&=\frac{\partial e}{\partial \xi_{i}}+p_{1}\frac{\left(m_{1}-m_{2}\right)\left(s-\frac{c_{V}+k}{\overline{m}}\right)+k\ln{\frac{\phi_{2}}{\phi_{2}}}}{\left(c_{V}+k\right)\rho}.
	\end{align}
\end{subequations}

% subsection eos (end)

\subsection{Perfect Gas} % (fold)
\label{sub:perfect}

Unfortunately, the form of Equation \ref{eq:pientropy} is exceedingly difficult to work with in a spectral sense.
In the assumption of a perfect gas, i.e. $de=c_{V}dT$, we can produce a more desirable system from Equations \ref{eq:firstlaw}, \ref{eq:pientropy}, and \ref{eq:picomposition}:
\begin{align}
	\frac{D}{Dt}e&=\frac{\partial e}{\partial s}\frac{D}{Dt}s+\frac{p}{\rho^{2}}\frac{D}{Dt}\rho+\sum_{i}\frac{\partial e}{\partial \xi_{i}}\frac{D}{Dt}\xi_{i}=c_{V}\frac{D}{Dt}T,\\
	c_{V}\rho\frac{D}{Dt}T&=\frac{1}{T}\frac{\partial e}{\partial s}\left(\nabla\vect{u}:\vect{\Pi}+\sum_{i}\mu_{i}\nabla\cdot\vect{C}_{i}-\nabla\cdot{\vect{H}}\right)+\nonumber\\
	&-p\nabla\cdot\vect{u}-\sum_{i}\frac{\partial e}{\partial \xi_{i}}\nabla\cdot\vect{C}_{i}.
\end{align}
At first glance, it doesn't appear that we've made much progress; however, we note that we can expand $\frac{1}{T}\frac{\partial e}{\partial s}$ into
\begin{equation}
	\frac{1}{T}\frac{\partial e}{\partial s}=\frac{1}{1+\frac{p_{1}}{T}\frac{\partial \frac{\partial e}{\partial s}/\partial\rho}{\partial p/\partial\rho}}=1-\frac{p_{1}}{T}\frac{\partial \frac{\partial e}{\partial s}/\partial\rho}{\partial p/\partial\rho}+O\left(p_{1}^{2}\right),
\end{equation}
which is a fine approximation since $p_1$ must be a small quantity for the pseudo-incompressible approximation to work at all.
A similar relation can be derived for $\mu_{i}$.
Using this, we can look at the temperature evolution equation up to first order in $p_{1}$:
\begin{align}
	c_{V}\rho\frac{D}{Dt}T&=\nabla\vect{u}:\vect{\Pi}-\nabla\cdot{\vect{H}}-p\nabla\cdot\vect{u}\nonumber\\
	&-\frac{p_{1}}{T}\frac{\partial \frac{\partial e}{\partial s}/\partial\rho}{\partial p/\partial\rho}\left(\nabla\vect{u}:\vect{\Pi}+\sum_{i}\frac{\partial e}{\partial \xi_{i}}\nabla\cdot\vect{C}_{i}-\nabla\cdot{\vect{H}}\right)+\nonumber\\
	&+\sum_{i}\frac{p_{1}}{\mu_{i}}\frac{\partial \frac{\partial e}{\partial \xi_{i}}/\partial\rho}{\partial p/\partial\rho}\nabla\cdot\vect{C}_{i}+O\left(p_{1}^{2}\right).
\end{align}
In this formulation, the strongest diffusive term is now a simple second derivative (buried in the divergence of the heat flux), which is much more conducive to a spectral solution.
All terms of order $p_{1}$ must be solved through nonlinear means.

% subsection perfect (end)

% section model (end)